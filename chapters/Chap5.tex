\chapter{Conclusion and Future Work}\label{chapter5}
\subsubsection{Conclusion}
In this thesis, two flux linkage estimators (ESO-FLE and IE-PU-FLE) were presented, which are applicable to any nonlinear synchronous machine.  Both estimators are designed as state observers in the time domain, eliminating the need for applying filters for estimation, thus avoiding any magnitude and phase distortions in the estimates. Additionally, ESO-FLE assumes nonlinear flux as ramp disturbance signals, while IE-PU-FLE uses RLS to estimate parameter variations online and compensates for parameter estimation errors with an adaptive observer, enabling both observers to handle parameter inaccuracies and improve transient estimation performance. Simulation results obtained using a 35-kW PMSM drive demonstrated that the proposed estimator closely tracked the true trajectories of the stator flux linkages under various operating conditions with better transient performance than the conventional estimators. 

\subsubsection{Future Work}

Future research will (i) focus on the optimal observer gain design to ensure stability and convergence considering parameter and speed variations (e.g., gain design based on LMI (Linear Matrix Inequality)), (ii) estimate flux linkage considering inverter nonlinearity (e.g., estimated based on Neural Network) or core losses, (iii) extract static or dynamic inductance based on online flux estimates (e.g., conducting optimal current control or MTPA algorithm), and finally, validate the performance and robustness of the proposed approach in the laboratory through experimental results.