
%%%%%%%%%%%%%%%%%%%%%%%%%%%%%%%%
% Chap 1. Introduction
%%%%%%%%%%%%%%%%%%%%%%%%%%%%%%%%

\chapter{Introduction}\label{chapter1}
\section{Motivation} \label{chap1:sec1}

Synchronous machines (SMs) have advantages such as high drive efficiency, output torque, power density, and excellent control performance \cite{c1_5},\cite{c2.1_1}. These advantages have led to their extensive use in various industries, including high-performance control applications such as servo systems \cite{c1_3} and home appliances \cite{c1_6}, as well as in renewable energy and eco-friendly sectors like wind power generators \cite{c1_4} and hybrid and electric vehicles \cite{c1_7},\cite{c1_8}.

Synchronous machines (SMs) are generally classified into three types: Permanent Magnet Synchronous Motor (PMSM), which has the same or different $d$-$q$ axis stator inductances and a rotor with embedded permanent magnets; Synchronous Reluctance Motor (synRM), which has different $d$-$q$ axis inductances due to the anisotropic rotor design that lacks permanent magnets and relies on rotor saliency for torque production; and Permanent Magnet-Assisted Synchronous Reluctance Motor (PMa-SynRM), which combines permanent magnets with a reluctance rotor to enhance torque production and efficiency. By appropriately utilizing the permanent magnets and the additional reluctance torque of these SMs, it is possible to maintain high output torque and efficiency over a wide operating range from low to high mechanical speeds. Therefore, due to these advantages, many automotive manufacturers have adopted SMs as traction motors for electric vehicles \cite{c1_7} (as shown in Table \ref{Table:1.1}).

\begin{table}[t]
\centering
\begin{tabular}{c|c|c|c|c|c}
\toprule
\hline
 & BMW i3 (2016) & Leaf2012 & Bosch SMG & Prius & McLaren\\
\hline
Motor type & IPM & IPM & PM & PM & SPM\\
\hline
Power density (kW/L) & 9.1 & 4.2 & 9.5 & 4.8 & -\\
\hline
Specific power (kW/kg) & 3.0 & 1.4 & 2.5 & 1.6 & 4.6\\
\hline
Max torque (Nm) & 250 & 207 & 198 & 207 & 130 \\
\hline
Max power (kW) & 125 & 80 & 80 & 60 & 120 \\
\hline
Max efficiency ($\%$) & 94 & 97 & 97 & 96 & 96 \\ 
\hline
\bottomrule
\end{tabular}
\caption{Specifications of EV traction motors in \cite{c1_7}.}
\label{Table:1.1}
\end{table}

In general, SMs generate torque through the interaction of the rotating magnetic fields produced by the stator and rotor, making stator flux linkage information crucial for controlling the torque of SMs. However, since flux cannot be directly measured by sensors, it is indirectly expressed using the machine parameters of the SMs (inductance, permanent magnet flux linkage) and the measurable stator current, allowing the SMs torque to be represented using these parameters and the current. 

These machine parameters of SMs are commonly used in fault diagnosis algorithms for detecting stator winding short circuits and demagnetization. These algorithms can enhance the reliability and efficiency of SMs by continuously monitoring them and predicting necessary maintenance tasks \cite{c1_9},\cite{c1_10},\cite{c1_11},\cite{c1_12},\cite{c1_13}. In addition, these parameters are also used in various optimal torque control strategies to minimize copper losses caused by the current in the stator windings, thereby increasing output efficiency and torque \cite{c1_2},\cite{c1_14}. In general, there are limitations to directly controlling the desired torque reference of SMs; it is necessary to convert the torque command into current references. However, since the output torque can be expressed as a bilinear function of the d- and q-axis currents, there are infinite combinations of d- and q-axis stator currents that can produce the same torque. Thus, these current combinations can be typically determined using the maximum torque per ampere (MTPA) control method, which maximizes torque per unit of current. However, when the mechanical speed of the synchronous machine exceeds the base speed, the increased back electromotive force (EMF) results in a limitation of the input voltage, making it impossible to maintain MTPA operation. Therefore, field weakening control is applied, which involves controlling the $d$-axis current to negative values to suppress the influence of the back EMF. Therefore, to optimally control the torque of the SMs and accurately implement these approaches, precise parameter information is ultimately essential. Moreover, the current references generated by the current reference generator can be optimally controlled through feedforward compensation-based current control or advanced optimal control based on model predictive control strategies \cite{c1_15},\cite{c1_16} such as Continuous Control Set Model Predictive Control (CCS-MPC) or Finite Control Set Model Predictive Control (FCS-MPC), for which accurate parameter or flux linkage information is necessary.

However, these parameters vary with the operating conditions of SMs, which are affected by the temperature of the stator windings, currents, and the temperature of the permanent magnets \cite{c1_17},\cite{c1_18}. To address these issues, several studies have presented adaptive-based methods for online estimation of multiple parameters \cite{c1_19},\cite{c1_20}, but these approaches fail to ensure estimation convergence. Some studies have suggested high-frequency current signal injection methods \cite{c2.3_3},\cite{c1_21} to temporarily resolve the rank deficiency issue and estimate parameters. Nevertheless, these approaches cause additional core losses reducing the efficiency of the synchronous machine, and fail to consider the inductance variation with $d$-$q$ axis current resulting in decreased accuracy of inductance estimation at the flux saturation region \cite{c1_22}. As a result of the absence of an accurate solution for online parameter estimation, the industry conventionally relies on 2-dimension lookup tables (LUTs) that list parameter information obtained through extensive experiments at all operating conditions \cite{c1_23},\cite{c1_25}. However using LUTs has several drawbacks because these tables require significant time and costly experimental resources, and interpolation must be used, which can degrade control performance due to errors with the actual parameters when operating the synchronous machine. Additionally, since LUTs are typically based on parameter information from a single motor sample, it is difficult to apply them uniformly to all synchronous machines due to parameter variations caused by manufacturing tolerances, aging, or temperature changes \cite{c1_24}.

In this background, a method for estimating parameters online without relying on LUTs is necessary to successfully operate SMs. As mentioned above, it is challenging to directly estimate the inductance and permanent magnet flux linkage parameters due to the rank deficiency of the estimation model. Instead of directly estimating these parameters, the stator flux linkage can be used, as it can be expressed as either the static inductance \cite{c2.1_3}, which has a linear relationship with the stator current, or the dynamic inductance \cite{c2.1_3},\cite{c1_26}, which has a nonlinear relationship with the stator current considering the magnetic saturation and cross-coupling effects. Thus, if the accurate stator flux linkage can be estimated online, it becomes possible to estimate these inductances online, making it necessary to first develop a high-performance stator flux linkage estimator.

\section{Research Objectives} \label{chap1:sec2}

Since the machine parameters of SMs continuously vary due to changes in temperature, speed, and current, a flux linkage estimator capable of online estimation of flux linkage or parameters is necessary to enhance the reliability and control performance of SMs. However, the estimation performance of existing flux estimators (especially transient estimation performance) is determined by the filters or nominal parameters used in the estimators. 

Therefore, the objectives of this study are (i) to develop an observable estimation model based on the stator flux linkage dynamics, (ii) to design a state observer in the time domain without relying on filters to improve estimation performance, and (iii) to develop a flux estimator that reduces dependence on nominal parameters to enhance estimation performance during transient states.

\section{Outline of the Thesis} \label{chap1:sec3}
This thesis is organized as follows: In Chapter \ref{chapter1}, the background and objectives of this study are discussed. In Chapter \ref{chapter2}, the mathematical modeling of synchronous machines (with interior permanent magnet synchronous machines as an exemplary model), optimal current control based on finite control set model predictive control, and existing research on flux linkage estimation are covered. In Chapter \ref{chapter3}, two flux estimators are proposed that can address the limitations of existing estimators by reducing parameter dependency and improving transient estimation performance. In Chapter \ref{chapter4}, the proposed flux estimators are validated through simulations and compared with the existing method. Finally, the conclusion of this study and future research are discussed in Chapter \ref{chapter5}, concluding the thesis.